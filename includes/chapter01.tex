% Comment this out if you're using the article class.
\chapter{Capítulo 1: La Introducción a los Salmos}

\section{Introducción}

Este salmo podría ser conocido como el prefacio del libro de los
Salmos. Nos provée una clara intrododucción sobre la senda del hombre
justo y la del hombre malo, su manera de andar y las consecuencias de
su proceder.

El salmista nos enseña el camino de la bendición; esto es el hombre
\textit{que no anduvo en consejo de malos ni en silla de
escarnecedores se ha sentado}. Pero tambien, nos enseña el camino a
la destrucción de aquel hombre que es \textit{como el tamo que arrebata el
viento} y que su final es perecer.

Existe una clara división en la expocision de este salmo que muestra
primeramente, la felicidad y bendición del hombre de Dios (vers. 1-3) y
posteriormente muestra el contraste del hombre malo y su final de
perdición (vers. 4-6).

\subsection{2 divisiones, el hombre de Dios vs el hombre malo}

\subsubsection{El camino del hombre de Dios}
\begin{quotebox}
\textbf{1} Bienaventurado el varón que no anduvo en consejo de malos,
ni estuvo en camino de pecadores, ni en silla de escarnecedores se ha
sentado;

\textbf{2} Sino que en la ley de Jehová esta su delicia, y en su ley
medita de dia y de noche.
\end{quotebox}

\textbf{\textit{Bienaventurado}} es una palabra que nos habla de
bendición y siempre antecede a una serie de beneficios para el sujeto del
cual se esta hablando.  Esto nos recuerda también el famoso Sermon del
Monte de Jesús en Mateo 5:3-14, donde Jesús nos habla de las
bienaventuranzas.

El verso 1 comienza hablando de la parte negativa en la cual el hombre
o mujer bienaventurado(a) no se han involucrado y en el verso 2 nos
habla de la elección que ha hecho de seguir los mandamientos de Dios,
puesto que: \textit{en la ley de Jehová esta su delicia}.

Sus pasos estan determinados por la palabra de Dios y no se deja
llevar por el \textbf{\textit{consejo de los malos}}.  Cuan vemos
\textit{no estuvo en camino de pecadores} nos habla de alguien que toma
determinación al decidir las compañias de las cuales se rodea. Aun en
condición pecadora, el bienaventurado ha sido lavado por sangre y
ahora con la ayuda del Espíritu Santo; su corazon es renovado y guiado
a rodearse de gente que le ayude en su crecimiento en el camino de
Dios.

\textbf{\textit{Ni en silla de escarnecedores se ha sentado}} nos
habla de la otra elección que ha hecho el hombre de Dios de no seguir
con las corrientes filosóficas de este mundo. Al contrario, el sigue
la plabra de Dios que es vida abundante y eterna. La silla de los
escarnecedores nos habla de un ambiente de constante blasfemia y
desobediencia a los estatutos establecidos en la palabra de Dios, un
lugar que esta cercano a la destrucción eterna.

\textbf{\textit{Sino que en la ley de Jehová esta su delicia}}. El
bienaventurado no vive en condenación, al contrario, el encuentra su
deleite en la palabra de Dios; y no solo eso, en ella
\textbf{\textit{medita de dia y de noche}}. Esto nos habla de alguien
que pone a Dios y su Palabra como prioridad en su vida y busca agradarle en
cada aspecto de su vida. Una constante búsqueda de su palabra nos
habla de dedicar tiempo y recursos al estudio de la palabra de Dios.

Esta tu delicida en la palabra de Dios? estudias la palabra de Dios?
meditas en ella de dia y de noche? Vivimos en un mundo donde el tiempo
muchas veces es considerado como dinero \textit{time is money}.
Resulta no redituable el invertir tiempo en leer un libro
que se escribió hace ya tanto tiempo. La Palabra de Dios es vida y
nunca deja de ser, El Autor sigue siendo el mismo de ayer hoy y
siempre. No hay duda que la palabra escrita por hombre guiados de
Dios sigue siendo vigente y de vital importancia aun en estos tiempos modernos.

\begin{quotebox}
  \textbf{3} Será como árbol plantado junto a corrientes de aguas, que
  da su fruto a su tiempo y su hoja no cae; y todo lo que hace,
  prosperará.
\end{quotebox}

\textbf{\textit{Será como arbol plantado}}, no cualquier arbol que
nace solo, un \textit{arbol plantado}. Escogido (1 Pedro 2:9-10),
considerado como propiedad, cultivado y cuidado del desarraigo (Mateo
15:13), toda planta que no ha sido plantada por el Padre, es destinada
al fracazo eterno.

\textbf{\textit{Junto a corrientes de aguas}} Podemos entendenderlo como:

\begin{itemize}
  \item Corrientes de la promesa.
  \item Corrientes de la Comunion con Cristo.
  \item Corrientes de Gracia.
  \item Corrientes de Perdon.
  \item e inumerable fuentes de provision y soporte por parte de Dios.
\end{itemize}

\textbf{\textit{Que da su fruto a su tiempo}} El hombre que se
deleita en la Palabra de Dios y es enseñado por ella recibe el gozo
de dar fruto. Dar fruto es algo esencial y que por naturaleza saldrá a
flote cuando encontrámos nuestro deleite en la Palabra de Dios y al
mismo tiempo buscámos cumplirla en nuestras vidas. Es importante
tambien conocer las conseciencias de no dar fruto. La Palabra de Dios
dice \textit{Todo árbol que no da buen fruto, es cortado y echado al fuego}
\textit{Mateo 7:19-23)}.

\begin{commentbox}{Fruto del Espíritu}
  \textbf{Gálatas 5:22-23}

  \textbf{22} Mas el fruto del Espíritu es amor, gozo, paz, paciencia,
  benignidad, bondad, fe, \textbf{23} mansedumbre, templanza; contra
  tales cosas no hay ley.
\end{commentbox}

\textbf{\textit{Y su hoja no cae}} expresa que no solo su fruto
estará presente y provisto a su tiempo, sino que tambien su hoja no
caerá. Su belleza, su resplandor y su frescura se mantendrán. El
hombre que encuentra su deleite en la Palabra de Dios, es alguien que
siempre tiene una buena respuesta ante cualquier circunstancia Sea
adversidad o de gozo siempre habrá una Palabra de Dios que le
respalde y le motive a seguir adelante.

\textbf{\textit{Y todo lo que hace prosperará}} Que bendición tan
tremenda es esta, una promesa que lo incluye “todo”. Su trabajo; la
obra de sus manos será prosperada (Deutoronomio 28:12) cuando pone en
primer lugar a Dios y Sus Estatutos (su Palabra). Aun cuando todo
parezca en contra, puede confiar en la promesa de Dios. Esto requiere
dar el paso de fe, caminar en fe y creyendo que el mismo Dios que ha
hecho esta promesa la cumplirá. Y aun cuando la prueba, la
persecución o la escasez vengan su confianza esta en Aquel dador de
todo y por El Cual todas las cosas creadas fueron hechas (Colosenses
1:16-17), sabiendo que a su debido tiempo la victoria vendrá.

\subsubsection{El camino del hombre malo}

\begin{quotebox}
  \textbf{4} No así los malos, que son como el tamo que arrebate el viento.
\end{quotebox}

Ahora comenzamos con el contraste del hombre malo, el hombre que no
reconoce los caminos de Dios. \textbf{\textit{No asi los malos}}
expresa que existe un presedente, los malos son todo lo contrario a lo
que el hombre bueno o hombre de Dios es.

\textbf{\textit{Son como el tamo}} que arrebate el viento, es decir,
son considerados como algo ligero, sin sustancia, muertos, inservibles
y facil de ser llevados de un lugar a
otro. \textbf{\textit{Arrebatados por el viento}} de corrientes
opuestas a las corrientes de agua viva en las que esta plantado el
hombre bueno.

\begin{quotebox}
  \textbf{5} Por tanto, no se levantarán los malos en el juicio, Ni
  los pecadores en la congregación de los justos.
\end{quotebox}

Los pecadores estarán en el dia del juicio, pero no para ser
justificados, sino para ser juzgados. No podrán levantar su frente y
no tendrán defensa alguna. \textbf{\textit{No se levantarán los malos
en el dia del jucio}}, al contrario serán echados fuera, al lago de
fuego.

No habrá pecado \textbf{\textit{en la congregación de los
justos}}. Los pecadores estarán fuera de la presencia de Dios. El
privilegio de estar en la congregación de los justos es solo para
aquellos que han decidido consagrarse a Dios.

\begin{quotebox}
  \textbf{6} Porque Jehová conoce el camino de los justos; mas la
  senda de los malos perecerá.
\end{quotebox}

El Señor \textbf{\textit{conoce el camino de los justos}}, El sabe y
prueba nuestros caminos \textit{(Job 23:10)}. \textbf{\textit{la senda
de los malos perecerá}}, un final terrible para aquellos que no
reconocen a Dios.

Conviene que sigámos los caminos de Dios, hay bendición y somos
bienaventurados al estar plantado junto a esas corrientes de agua
viva. Dios conoce nuestros caminos \textit{(Jeremías 17:10)} y dará
conforme a la decisión que hallamos tomado. El fruto es claro, unirte a
la congregación de los justos o perecer por siempre con los malos.

\begin{paperbox}{Un llamado a la acción}
  \textit{Llamado a la acción} es una sección que te presentará
  desafios, reflexiones y palabras que te lleven a la acción y  a llevar a
  cabo aquello que Dios te esta llamando a hacer. Todo esto para a cumplir su
  voluntad en tu vida y tu ministerio.
\end{paperbox}


\begin{monsterbox}{Escudriñando ... }
  \textit{Cada lección tendrá una sección llamada
    \textbf{Escudriñando}, la cual contendrá  ejercios para
    reforzar los aprendido en cada lección. A continuación los primeros
    2 ejercicios}
  \hline%
  Memorizar el Samo 1.
  \hline%
  Leer Salmo 2.
  \vspace{5mm}
\end{monsterbox}
